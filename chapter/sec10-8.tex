\subsection{特征值与特征向量}
	\begin{titwo}
		已知 $\bm \alpha_{1} = [-1,1,a,4]^{\TT}$, $\bm \alpha_{2} = [-2,1,5,a]^{\TT}$, $\bm \alpha_{3} = [a,2,10,1]^{\TT}$ 是 $4$ 阶方阵 $\bm A$ 的 $3$ 个不同特征值对应的特征向量,则 $a$ 的取值范围为 \kuo.

		\twoch{$a \ne 5$}{$a \ne -4$}{$a \ne -3$}{$a \ne -3$ 且 $a \ne -4$}
	\end{titwo}

	\begin{titwo}
		设 $\bm A$, $\bm B$ 为 $n$ 阶矩阵,且 $\bm A$ 与 $\bm B$ 相似,$\bm E$为 $n$ 阶单位矩阵,则 \kuo.

		\onech{$\lambda \bm E - \bm A = \lambda \bm E - \bm B$}{$\bm A$ 与 $\bm B$ 有相同的特征值和特征向量}{$\bm A$ 与 $\bm B$ 都相似于一个对角矩阵}{对任意常数 $t$, $t \bm E - \bm A$ 与 $t \bm E - \bm B$ 相似}
	\end{titwo}

	\begin{titwo}
		已知 $3$ 阶矩阵 $\bm A$ 有特征值 $\lambda_{1} = 1$, $\lambda_{2} = 2$, $\lambda_{3} = 3$,则 $2 \bm A^{\astt}$ 的特征值是 \kuo.

		\fourch{$1,2,3$}{$4,6,12$}{$2,4,6$}{$8,16,24$}
	\end{titwo}

	\begin{titwo}
		已知 $\bm \xi_{1}$, $\bm \xi_{2}$ 是方程 $(\lambda E - \bm A) \bm X = \bm 0$ 的两个不同的解向量,则下列向量中必是 $\bm A$ 的对应于特征值 $\lambda$ 的特征向量的是 \kuo.

		\fourch{$\bm \xi_{1}$}{$\bm \xi_{2}$}{$\bm \xi_{1} - \bm \xi_{2}$}{$\bm \xi_{1} + \bm \xi_{2}$}
	\end{titwo}

	\begin{titwo}
		设
		\[
			\bm A = \begin{bsmallmatrix}
				-1 & 2 & 3 \\
				2 & -1 & 0 \\
				3 & 3 & 1
			\end{bsmallmatrix},
		\]
		则下列选项中是 $\bm A$ 的特征向量的是 \kuo.

		\twoch{$\bm \xi_{1} = [1,2,1]^{\TT}$}{$\bm \xi_{2} = [1,-2,1]^{\TT}$}{$\bm \xi_{3} = [2,1,2]^{\TT}$}{$\bm \xi_{4} = [2,1,-2]^{\TT}$}
	\end{titwo}

	\begin{titwo}
		已知 $\bm A$, $\bm B$ 为 $3$ 阶相似矩阵,$\lambda_{1} = 1$, $\lambda_{2} = 2$ 为 $\bm A$ 的两个特征值,$|\bm B| = 2$,则行列式 $\begin{vsmallmatrix}
			(\bm A + \bm E)^{-1} & \bm O \\
			\bm O & (2 \bm B)^{\astt}
		\end{vsmallmatrix} = $ \htwo.
	\end{titwo}

	\begin{titwo}
		已知 $-2$ 是 $\bm A = \begin{bsmallmatrix}
			0 & -2 & -2 \\
			2 & x & -2 \\
			-2 & 2 & b
		\end{bsmallmatrix}$ 的特征值,其中 $b$ ($b \ne 0$) 是任意常数,则 $x = $ \htwo.
	\end{titwo}

	\begin{titwo}
		设 $\bm A$ 是 $3$ 阶矩阵,$|\bm A| = 3$,且满足 $|\bm A^{2} + 2 \bm A| = 0$,$|2 \bm A^{2} + \bm A| = 0$,则 $\bm A^{\astt}$ 的特征值是 \htwo.
	\end{titwo}

	\begin{titwo}
		设 $\bm A$ 是 $n$ 阶实对称矩阵,$\lambda_{1}$, $\lambda_{2}$, $\cdots$, $\lambda_{n}$ 是 $\bm A$ 的 $n$ 个互不相同的特征值,$\bm \xi_{1}$ 是 $\bm A$ 的对应于 $\lambda_{1}$ 的一个单位特征向量,则矩阵 $\bm B = \bm A - \lambda_{1} \bm \xi_{1} \bm \xi_{1}^{\TT}$ 的特征值是 \htwo.
	\end{titwo}

	\begin{titwo}
		设 $\bm A$ 是 $3$ 阶矩阵,$\bm \xi_{1}$, $\bm \xi_{2}$, $\bm \xi_{3}$ 是三个线性无关的 $3$ 维列向量,满足 $\bm A \bm \xi_{i} = \bm \xi_{i}$, $i = 1,2,3$,则 $\bm A = $ \htwo.
	\end{titwo}

	\begin{titwo}
		设 $\bm A$ 为 $n$ 阶矩阵,$\lambda_{1}$ 和 $\lambda_{2}$ 是 $\bm A$ 的两个不同的特征值,$\bm x_{1}$, $\bm x_{2}$ 是分别属于 $\lambda_{1}$ 和 $\lambda_{2}$ 的特征向量. 证明:$\bm x_{1} + \bm x_{2}$ 不是 $\bm A$ 的特征向量.
	\end{titwo}

	\begin{titwo}
		已知 $\bm \alpha = [1,k,1]^{\TT}$ 是 $\bm A^{-1}$ 的特征向量,其中 $\bm A = \begin{bsmallmatrix}
			2 & 1 & 1 \\
			1 & 2 & 1 \\
			1 & 1 & 2
		\end{bsmallmatrix}$,求 $k$ 及 $\bm \alpha$ 所对应的 $\bm A$ 的特征值.
	\end{titwo}

	\begin{titwo}
		设 $\bm A$ 是 $n$ 阶方阵,$2,4,\cdots,2n$ 是 $\bm A$ 的 $n$ 个特征值,$\bm E$ 是 $n$ 阶单位阵,计算行列式 $|\bm A - 3 \bm E|$ 的值.
	\end{titwo}

	\begin{titwo}
		设 $\lambda_{1}, \lambda_{2}$ 是 $n$ 阶矩阵 $\bm A$ 的特征值,$\bm \alpha_{1}, \bm \alpha_{2}$ 分别是 $\bm A$ 的对应于 $\lambda_{1}, \lambda_{2}$ 的特征向量,则 \kuo.

		\onech{当 $\lambda_{1} = \lambda_{2}$ 时,$\bm \alpha_{1}, \bm \alpha_{2}$ 对应分量必成比例}{当 $\lambda_{1} = \lambda_{2}$ 时,$\bm \alpha_{1}, \bm \alpha_{2}$ 对应分量不成比例}{当 $\lambda_{1} \ne \lambda_{2}$ 时,$\bm \alpha_{1}, \bm \alpha_{2}$ 对应分量必成比例}{当 $\lambda_{1} \ne \lambda_{2}$ 时,$\bm \alpha_{1}, \bm \alpha_{2}$ 对应分量必不成比例}
	\end{titwo}

	\begin{titwo}
		设 $\bm A$ 为 $n$ 阶矩阵,则下列命题正确的是 \kuo.

		\onech{若 $\bm \alpha$ 为 $\bm A^{\TT}$ 的特征向量,那么 $\bm \alpha$ 为 $\bm A$ 的特征向量}{若 $\bm \alpha$ 为 $\bm A^{\astt}$ 的特征向量,那么 $\bm \alpha$ 为 $\bm A$ 的特征向量}{若 $\bm \alpha$ 为 $\bm A^{2}$ 的特征向量,那么 $\bm \alpha$ 为 $\bm A$ 的特征向量}{若 $\bm \alpha$ 为 $2 \bm A$ 的特征向量,那么 $\bm \alpha$ 为 $\bm A$ 的特征向量}
	\end{titwo}

	\begin{titwo}
		已知 $\bm A$ 是 $3$ 阶矩阵,$r(\bm A) = 1$,则 $\lambda = 0$ \kuo.

		\onech{必是 $\bm A$ 的二重特征值}{至少是 $\bm A$ 的二重特征值}{至多是 $\bm A$ 的二重特征值}{一重、二重、三重特征值都可能}
	\end{titwo}

	\begin{titwo}
		$\bm A$ 是 $n$ 阶矩阵,则 $\bm A$ 相似于对角矩阵的充分必要条件是 \kuo.

		\onech{$\bm A$ 有 $n$ 个不同的特征值}{$\bm A$ 有 $n$ 个不同的特征向量}{对 $\bm A$ 的每个 $r_{i}$ 重特征值 $\lambda_{i}$,都有 $r(\lambda_{i} \bm E - \bm A) = n - r_{i}$}{$\bm A$ 是实对称矩阵}
	\end{titwo}

	\begin{titwo}
		已知 $\bm P^{-1} \bm A \bm P = \begin{bsmallmatrix}
			2 & 0 & 0 \\
			0 & 6 & 0 \\
			0 & 0 & 6
		\end{bsmallmatrix}$,$\bm \alpha_{1}$ 是矩阵 $\bm A$ 属于特征值 $\lambda = 2$ 的特征向量,$\bm \alpha_{2}, \bm \alpha_{3}$ 是矩阵 $\bm A$ 属于特征值 $\lambda = 6$ 的线性无关的特征向量,那么矩阵 $\bm P$ 不能是 \kuo.

		\twoch{$[ \bm \alpha_{1}, - \bm \alpha_{2}, \bm \alpha_{3} ]$}{$[ \bm \alpha_{1}, \bm \alpha_{2} + \bm \alpha_{3}, \bm \alpha_{2} - 2 \bm \alpha_{3} ]$}{$[ \bm \alpha_{1}, \bm \alpha_{3}, \bm \alpha_{2} ]$}{$[\bm \alpha_{1} + \bm \alpha_{2}, \bm \alpha_{1} - \bm \alpha_{2}, \bm \alpha_{3}]$}
	\end{titwo}

	\begin{titwo}
		设 $\bm A, \bm B$ 为 $3$ 阶相似矩阵,且 $|2 \bm E + \bm A| = 0$,$\lambda_{1} = 1,$ $\lambda_{2} = -1$ 为 $\bm B$ 的两个特征值,则行列式 $|\bm A + 2 \bm A \bm B| = $ \htwo.
	\end{titwo}

	\begin{titwo}
		设 $\bm A = \bm E + \bm \alpha \bm \beta^{\TT}$,其中 $\bm \alpha, \bm \beta$ 均为 $n$ 维列向量,$\bm \alpha^{\TT}  \bm \beta = 3$,则 $|\bm A + 2 \bm E| = $ \htwo.
	\end{titwo}

	\begin{titwo}
		矩阵 $\bm A = \begin{bsmallmatrix}
			1 & 1 & 1 & 1 \\
			1 & 1 & 1 & 1 \\
			1 & 1 & 1 & 1 \\
			1 & 1 & 1 & 1
		\end{bsmallmatrix}$ 的非零特征值是 \htwo.
	\end{titwo}

	\begin{titwo}
		设 $\bm A$ 是 $3$ 阶矩阵,已知 $|\bm A + \bm E| = 0$,$|\bm A + 2 \bm E| = 0$,$|\bm A + 3 \bm E| = 0$,则 $|\bm A + 4 \bm E| = $ \htwo.
	\end{titwo}

	\begin{titwo}
		设 $n$ 阶矩阵 $\bm A$ 的元素全是 $1$,则 $\bm A$ 的 $n$ 个特征值是 \htwo.
	\end{titwo}

	\begin{titwo}
		已知 $\bm \alpha = [\alpha,1,1]^{\TT}$ 是矩阵 $\bm A = \begin{bsmallmatrix}
			-1 & 2 & 2 \\
			2 & a & -2 \\
			2 & -2 & -1
		\end{bsmallmatrix}$ 的逆矩阵的特征向量,那么 $a = $ \htwo.
	\end{titwo}

	\begin{titwo}
		\begin{enumerate}
			\item 设 $\lambda_{1},\lambda_{2},\cdots,\lambda_{n}$ 是 $n$ 阶矩阵 $\bm A$ 的互异特征值,$\bm \alpha_{1},\bm \alpha_{2},\cdots,\bm \alpha_{n}$ 是 $\bm A$ 的分别对应于这些特征值的特征向量,证明 $\bm \alpha_{1},\bm \alpha_{2},\cdots,\bm \alpha_{n}$ 线性无关;
			\item 设 $\bm A, \bm B$ 为 $n$ 阶方阵,$|\bm B| \ne 0$,若方程 $|\bm A - \lambda \bm B| = 0$ 的全部根 $\lambda_{1},\lambda_{2},\cdots,\lambda_{n}$ 互异,$\bm \alpha_{i}$ 分别是方程组 $(\bm A - \lambda_{i} \bm B) \bm x = \bm 0$ 的非零解,$i = 1,2,\cdots,n$. 证明 $\bm \alpha_{1},$ $\bm \alpha_{2},\cdots,\bm \alpha_{n}$ 线性无关.
		\end{enumerate}
	\end{titwo}

	\begin{titwo}
		设 $\bm A$ 是 $n \times n$ 矩阵,对任何 $n$ 维列向量 $\bm X$ 都有 $\bm A \bm X = \bm 0$,证明:$\bm A = \bm O$.
	\end{titwo}

	\begin{titwo}
		设有 $4$ 阶方阵 $\bm A$ 满足条件 $|3 \bm E + \bm A| = 0$,$\bm A \bm A^{\TT} = 2 \bm E$,$|\bm A| < 0$,其中 $\bm E$ 是 $4$ 阶单位矩阵. 求方阵 $\bm A$ 的伴随矩阵 $\bm A^{\astt}$ 的一个特征值.
	\end{titwo}

	\begin{titwo}
		已知 $\bm B$ 是 $n$ 阶矩阵,满足 $\bm B^{2} = \bm E$ (此时矩阵 $\bm B$ 称为对合矩阵). 求 $\bm B$ 的特征值的取值范围.
	\end{titwo}

	\begin{titwo}
		设 $\bm A, \bm B$ 是 $n$ 阶方阵,证明:$\bm A \bm B, \bm B \bm A$ 有相同的特征值.
	\end{titwo}

	\begin{titwo}
		已知 $n$ 阶矩阵 $\bm A$ 的每行元素之和为 $a$,求 $\bm A$ 的一个特征值. 当 $k$ 是自然数时,求 $\bm A^{k}$ 的每行元素之和.
	\end{titwo}

	\begin{titwo}
		设 $\bm A$ 是 $3$ 阶矩阵,$\lambda_{1},\lambda_{2},\lambda_{3}$ 是三个不同的特征值,$\bm \xi_{1},\bm \xi_{2},\bm \xi_{3}$ 是相应的特征向量. 证明:向量组 $\bm A(\bm \xi_{1} + \bm \xi_{2})$, $\bm A(\bm \xi_{2} + \bm \xi_{3})$, $\bm A(\bm \xi_{3} + \bm \xi_{1})$ 线性无关的充要条件是 $\bm A$ 是可逆矩阵.
	\end{titwo}

	\begin{titwo}
		设 $\bm A$ 是 $3$ 阶实矩阵,$\lambda_{1},\lambda_{2},\lambda_{3}$ 是 $\bm A$ 的三个不同的特征值,$\bm \xi_{1},\bm \xi_{2},\bm \xi_{3}$ 是三个对应的特征向量. 证明:当 $\lambda_{2} \lambda_{3} \ne 0$ 时,向量组 $\bm \xi_{1},\bm A (\bm \xi_{1} + \bm \xi_{2}),\bm A^{2} (\bm \xi_{1} + \bm \xi_{2} + \bm \xi_{3})$ 线性无关.
	\end{titwo}

	\begin{titwo}
		设 $\bm A$ 是 $n$ 阶实矩阵,有 $\bm A \bm \xi = \lambda \bm \xi$, $\bm A^{\TT} \bm \eta = \mu \bm \eta$,其中 $\lambda$, $\mu$ 是实数,且 $\lambda \ne \mu$,$\bm \xi$, $\bm \eta$ 是 $n$ 维非零向量. 证明:$\bm \xi$, $\bm \eta$ 正交.
	\end{titwo}

	\begin{titwo}
		设矩阵 $\bm A = \begin{bsmallmatrix}
			a & -1 & c \\
			5 & b & 3 \\
			1 - c & 0 & -a
		\end{bsmallmatrix}$,且 $|\bm A| = -1$,$\bm A$ 的伴随矩阵 $\bm A^{\astt}$ 有特征值 $\lambda_{0}$,属于 $\lambda_{0}$ 的特征向量维 $\bm \alpha = [-1,-1,1]^{\TT}$,求 $a$, $b$, $c$ 及 $\lambda_{0}$ 的值.
	\end{titwo}

	\begin{titwo}
		设 $\bm A$ 是 $3$ 阶实对称矩阵,$\lambda_{1} = -1$, $\lambda_{2} = \lambda_{3} = 1$ 是 $\bm A$ 的特征值,对应于 $\lambda_{1}$ 的特征向量为 $\bm \xi_{1} = [0,1,1]^{\TT}$,求 $\bm A$.
	\end{titwo}

	\begin{titwo}
		设 $\bm A$ 为 $3$ 阶矩阵,$\lambda_{1}$, $\lambda_{2}$, $\lambda_{3}$ 是 $\bm A$ 的三个不同特征值,对应的特征向量为 $\bm \alpha_{1}$, $\bm \alpha_{2}$, $\bm \alpha_{3}$,令 $\bm \beta = \bm \alpha_{1} + \bm \alpha_{2} + \bm \alpha_{3}$.
		\begin{enumerate}
			\item 证明 $\bm \beta$, $\bm A \bm \beta$, $\bm A^{2} \bm \beta$ 线性无关;
			\item 若 $\bm A^{3} \bm \beta = \bm A \bm \beta$,求秩 $r(\bm A - \bm E)$ 及行列式 $|\bm A + 2 \bm E|$.
		\end{enumerate}
	\end{titwo}

	\begin{titwo}
		设 $\bm A$ 是 $3$ 阶矩阵,$\lambda_{1} = 1$, $\lambda_{2} = 2$, $\lambda_{3} = 3$ 是 $\bm A$ 的特征值,对应的特征向量分别是
		\[
			\bm \xi_{1} = [2,2,-1]^{\TT}, \bm \xi_{2} = [-1,2,2]^{\TT}, \bm \xi_{3} = [2,-1,2]^{\TT}.
		\]
		又 $\bm \beta = [1,2,3]^{\TT}$. 计算:
		\begin{enumerate}
			\item $\bm A^{n} \bm \xi_{1}$;
			\item $\bm A^{n} \bm \beta$.
		\end{enumerate}
	\end{titwo}

	\begin{titwo}
		设 $\bm A$ 是 $2$ 阶实对称矩阵,有特征值 $\lambda_{1} = 4$, $\lambda_{2} = -1$, $\bm \xi_{1} = [-2,1]^{\TT}$ 是 $\bm A$ 对应于 $\lambda_{1}$ 的特征向量,$\bm \beta = [3,1]^{\TT}$,则 $\bm A \bm \beta = $ \htwo.
	\end{titwo}

	\begin{titwo}
		设 $\bm A$ 是 $3$ 阶实对称矩阵,已知 $\bm A$ 的每行元素之和为 $3$,且有二重特征值 $\lambda_{1} = \lambda_{2} = 1$. 求 $\bm A$ 的全部特征值、特征向量,并求 $\bm A^{n}$.
	\end{titwo}

	\begin{titwo}
		设 $\bm A$, $\bm B$ 均为 $n$ 阶矩阵,$\bm A$ 可逆且 $\bm A \sim \bm B$,则下列命题中:\circled{1} $\bm A \bm B \sim \bm B \bm A$; \circled{2} $\bm A^{2} \sim \bm B^{2}$; \circled{3} $\bm A^{\TT} \sim \bm B^{\TT}$; \circled{4} $\bm A^{-1} \sim \bm B^{-1}$. 正确命题的个数为 \kuo.

		\fourch{1}{2}{3}{4}
	\end{titwo}

	\begin{titwo}
		设矩阵 $\bm A = \begin{bsmallmatrix}
			3 & 2 & -2 \\
			-k & -1 & k \\
			4 & 2 & -3
		\end{bsmallmatrix}$,问 $k$ 为何值时,存在可逆矩阵 $\bm P$,使得 $\bm P^{-1} \bm A \bm P = \bm \varLambda$,求出 $\bm P$ 及相应的对角矩阵.
	\end{titwo}

	\begin{titwo}
		设矩阵 $\bm A = \begin{bsmallmatrix}
			1 & -1 & 1 \\
			x & 4 & y \\
			-3 & -3 & 5
		\end{bsmallmatrix}$ 有三个线性无关的特征向量,$\lambda = 2$ 是 $\bm A$ 的二重特征值,试求可逆矩阵 $\bm P$ 使得 $\bm P^{-1} \bm A \bm P = \bm \varLambda$,其中 $\bm \varLambda$ 是对角矩阵.
	\end{titwo}

	\begin{titwo}
		已知 $\bm \xi = [1,1,-1]^{\TT}$ 是矩阵 $\bm A = \begin{bsmallmatrix}
			2 & -1 & 2 \\
			5 & a & 3 \\
			-1 & b & -2
		\end{bsmallmatrix}$ 的一个特征向量.
		\begin{enumerate}
			\item 确定参数 $a$, $b$ 及 $\bm \xi$ 对应的特征值 $\lambda$;
			\item $\bm A$ 是否相似于对角矩阵,说明理由.
		\end{enumerate}
	\end{titwo}