\subsection{中值定理、方程的根、不等式}

	\begin{ti}
		设函数 $f(x)$ 在 $[a,b]$ 上连续,在 $(a,b)$ 内可导,且 $f(a) = f(b) = 0$,求证:
		\begin{enumerate}
			\item 存在 $\xi \in (a,b)$,使 $f(\xi) + \xi f'(\xi) = 0$;
			\item 存在 $\eta \in (a,b)$,使 $\eta f(\eta) + f'(\eta) = 0$.
		\end{enumerate}
	\end{ti}

	\begin{ti}
		设函数 $f(x)$ 在 $[-2,2]$ 上二阶可导,且 $\left| f(x) \right| \leq 1$,又 $f^{2}(0) + \left[ f'(0) \right]^{2} = 4$. 试证:在 $(-2,2)$ 内至少存在一点 $\xi$,使得 $f(\xi) + f''(\xi) = 0$.
	\end{ti}

	\begin{ti}
		设函数 $f(x)$ 在 $[a,b]$ 上连续,在 $(a,b)$ 内可导且 $f(a) \ne f(b)$. 证明:存在 $\xi,\eta \in (a,b)$,使得 $\frac{f'(\xi)}{2\xi} = \frac{f'(\eta)}{b + a}$.
	\end{ti}

	\begin{ti}
		设函数 $f(x)$ 在闭区间 $[a,b]$ 上连续 $(a,b > 0)$,在 $(a,b)$ 内可导. 证明:在 $(a,b)$ 内至少存在一点 $\xi$,使等式 $\frac{1}{a - b} \left| \begin{smallmatrix}
			a & b\\
			f(a) & f(b)
		\end{smallmatrix} \right| = f(\xi) - \xi f'(\xi)$ 成立.
	\end{ti}

	\begin{ti}
		设 $f(x)$ 在闭区间 $[1,2]$ 上可导,证明:存在一点 $\xi \in (1,2)$,使
		\[
			f(2) - 2f(1) = \xi f'(\xi) - f(\xi).
		\]
	\end{ti}

	\begin{ti}
		设 $f(x), g(x)$ 在 $[a,b]$ 上二阶可导,且 $f(a) = f(b) = g(a) = 0$,证明:存在 $\xi \in (a,b)$,使 $f''(\xi) g(\xi) + 2 f'(\xi) g'(\xi) + f(\xi) g''(\xi) = 0$.
	\end{ti}

	\begin{ti}
		设 $f(x)$ 在闭区间 $[-1,1]$ 上具有三阶连续导数,且 $f(-1) = 0$,$f(1) = 1$,$f'(0) = 0$. 证明:在 $[-1,1]$ 内存在 $\xi$,使得 $f'''(\xi) = 3$.
	\end{ti}

	\begin{ti}
		设 $f(x), g(x)$ 在 $[a,b]$ 上二阶可导,$g''(x) \ne 0$,$f(a) = f(b) = g(a) = g(b) = 0$. 证明:
		\begin{enumerate}
			\item 在 $(a,b)$ 内,$g(x) \ne 0$;
			\item 在 $(a,b)$ 内至少存在一点 $\xi$,使 $\frac{f(\xi)}{g(\xi)} = \frac{f''(\xi)}{g''(\xi)}$.
		\end{enumerate}
	\end{ti}

	\begin{ti}
		$f(x)$ 在 $[a,b]$ 上连续,在 $(a,b)$ 内可导,且 $f'(x) \ne 0$. 证明:存在 $\xi, \eta \in (a,b)$,使得
		\[
			\frac{f'(\xi)}{f'(\eta)} = \frac{\ee^{b} - \ee^{a}}{b - a} \ee^{-\eta}.
		\]
	\end{ti}

	\begin{ti}
		设 $f(x)$ 在 $[0,1]$ 上可导,$f(0) = 0$,$\left| f'(x) \right| \leq \left| f(x) \right|$,则 $f(1) = $\htwo.
	\end{ti}

	\begin{ti}
		设 $f(x)$ 在 $[0,4]$ 上一阶可导且 $f'(x) \geq \frac{1}{4}$,$f(2) \geq 0$,则在下列区间上必有 $f(x) \geq \frac{1}{4}$ 成立的是\kuo.

		\fourch{$[0,1]$}{$[1,2]$}{$[2,3]$}{$[3,4]$}
	\end{ti}

	\begin{ti}
		设 $f(x)$ 二阶可导,$f''(x) < 0$,$f'(0) \leq \frac{1}{3}$,$f(0) = 0$,$f(1) = \frac{1}{2}$,并设 $0 < x_{n} < 1$,且 $x_{n+1} = f(x_{n}), n = 1,2,\cdots$.
		\begin{enumerate}
			\item 证明 $\frac{f(x)}{x}$ 在 $(0,+\infty)$ 内单调减少;
			\item 证明 $\lim_{n \to \infty} x_{n}$ 存在.
		\end{enumerate}
	\end{ti}

	\begin{ti}
		设 $\xi_{a}$ 为函数 $f(x) = \arctan x$ 在区间 $[0,a]$ 上使用拉格朗日中值定理时的中值,求 $\lim_{a \to 0^{+}} \frac{\xi_{a}}{a}$.
	\end{ti}

	\begin{ti}
		设函数 $f(x) = \arctan x$,若 $f(x) = f'(\xi) \sin x$,求极限 $\lim_{x \to 0} \frac{\xi^{2}}{x^{2}}$.
	\end{ti}

	\begin{ti}
		设函数 $f(x)$ 在闭区间 $[a,b]$ 上连续,在开区间 $(a,b)$ 内可导,$f'(x) \ne 0$,且 $f(a) = 0$,$f(b) = 2$. 证明在开区间 $(a,b)$ 内存在两个不同的点 $\xi,\eta$,使
		\[
			f'(\eta) \left[ f(\xi) + \xi f'(\xi) \right] = f'(\xi) \left[ b f'(\eta) - 1 \right].
		\]
	\end{ti}

	\begin{ti}
		讨论常数 $a$ 的值,确定曲线 $y = a \ee^{x}$ 与 $y = 1 + x$ 的公共点的个数.
	\end{ti}

	\begin{ti}
		讨论方程 $2x^{3} - 9x^{2} + 12x - a = 0$ 实根的情况.
	\end{ti}

	\begin{ti}
		讨论方程 $a x \ee^{x} + b = 0 (a > 0)$ 实根的情况.
	\end{ti}

	\begin{ti}
		证明:方程 $x^{\alpha} = \ln x (\alpha < 0)$ 在 $(0,+\infty)$ 内有且仅有一个实根.
	\end{ti}

	\begin{ti}
		设 $f(x)$ 可导,证明:$f(x)$ 的两个零点之间一定有 $f(x) + f'(x)$ 的零点.
	\end{ti}

	\begin{ti}
		设 $F(x) = \int_{-1}^{1} |x - t| \ee^{-t^{2}} \dd{t} - \frac{1}{2} \left( \ee^{-1} + 1 \right)$,讨论 $F(x)$ 在区间 $[-1,1]$ 上的零点个数.
	\end{ti}

	\begin{ti}
		证明:当 $x \geq 1$ 时,$\arctan x - \frac{1}{2} \arccos \frac{2x}{1 + x^{2}} \equiv \frac{\uppi}{4}$.
	\end{ti}

	\begin{ti}
		设 $\lim_{x \to 0} \frac{f(x)}{x} = 1$,且 $f''(x) > 0$. 证明:$f(x) \geq x$.
	\end{ti}

	\begin{ti}
		证明:当 $0 < a < b < \uppi$ 时,
		\[
			b \sin b + 2 \cos b + \uppi b > a \sin a + 2 \cos a + \uppi a.
		\]
	\end{ti}

	\begin{ti}
		设 $b > a > \ee$,证明:$a^{b} > b^{a}$.
	\end{ti}

	\begin{ti}
		设一质点在单位时间内由点 $A$ 从静止开始做直线运动至点 $B$ 停止,$A,B$ 两点间距离为 $1$,证明:该质点在 $(0,1)$ 内总有一时刻的加速度的绝对值不小于 $4$.
	\end{ti}

	\begin{ti}
		在区间 $[0,a]$ 上 $\left| f''(x) \right| \leq M$,且 $f(x)$ 在 $(0,a)$ 内取得最大值. 求证:$\left| f'(0) \right| + \left| f'(a) \right| \leq Ma$.
	\end{ti}

	\begin{ti}
		设 $x \in (0,1)$,证明下面不等式
		\begin{enumerate}
			\item $(1 + x) \ln^{2}(1 + x) < x^{2}$;
			\item $\frac{1}{\ln 2} - 1 < \frac{1}{\ln(1 + x)} - \frac{1}{x} < \frac{1}{2}$.
		\end{enumerate}
	\end{ti}

	\begin{ti}
		证明:$\cos \sqrt{2} x \leq - x^{2} + \sqrt{1 + x^{4}}$,其中 $x \in \bigl( 0, \frac{\sqrt{2}\uppi}{4} \bigr)$.
	\end{ti}

	\begin{ti}
		设 $f(x)$ 在 $[a,b]$ 上二阶可导,且 $f'(a) = f'(b) = 0$,证明存在 $\xi \in (a,b)$,使
		\[
			\left| f''(\xi) \right| \geq \frac{4}{(b - a)^{2}} \left| f(b) - f(a) \right|.
		\]
	\end{ti}

	\begin{ti}
		已知 $f(x)$ 二阶可导,且 $f(x) > 0$,$f(x) f''(x) - \bigl[ f'(x) \bigr]^{2} \geq 0 (x \in \mathbb{R})$.
		\begin{enumerate}
			\item 证明 $f(x_{1}) f(x_{2}) \geq f^{2}\left( \frac{x_{1} + x_{2}}{2} \right)(x_{1}, x_{2} \in \mathbb{R})$;
			\item 若 $f(0) = 1$,证明 $f(x) \geq \ee^{f'(0) x} (x \in \mathbb{R})$.
		\end{enumerate}
	\end{ti}

	\begin{ti}
		设 $f(x)$ 在 $[a,b]$ 上具有二阶导数,且 $f''(x) > 0$,证明:
		\[
			f\left( \frac{a + b}{2} \right) < \frac{1}{b - a} \int_{a}^{b} f(t) \dd{t} < \frac{1}{2} \left[ f(a) + f(b) \right].
		\]
	\end{ti}

	\begin{ti}
		证明:$\ee^{x} + \ee^{-x} \geq 2x^{2} + 2 \cos x, -\infty < x < +\infty$.
	\end{ti}

	\begin{ti}
		设 $f(x)$ 在区间 $[0,+\infty)$ 内具有二阶导数,且 $\bigl|f(x)\bigr| \leq 1, 0 < \bigl| f''(x) \bigr| \leq 2 (0 \leq x < +\infty)$. 证明:$\bigl| f'(x) \bigr| \leq 2\sqrt{2}$.
	\end{ti}

	\begin{ti}
		若用 $\frac{2(x - 1)}{x + 1}$ 来近似 $\ln x$,证明当 $x \in [1,2]$ 时,其误差不超过 $\frac{1}{12} (x - 1)^{3}$.
	\end{ti}