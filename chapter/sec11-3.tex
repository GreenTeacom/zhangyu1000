\section{二维随机变量及其分布}
	\begin{titwo}
		已知随机变量 $(X_{1},X_{2})$ 的概率密度为 $f_{1}(x_{1},x_{2})$,设 $Y_{1} = 2X_{1}$, $Y_{2} = \frac{1}{3} X_{2}$,则随机变量 $(Y_{1},Y_{2})$ 的概率密度 $f_{2}(y_{1},y_{2}) = $ \kuo.

		\twoch{$f_{1} \bigl( \frac{y_{1}}{2}, 3y_{2} \bigr)$}{$\frac{3}{2} f_{1} \bigl( \frac{y_{1}}{2}, 3y_{2} \bigr)$}{$f_{1} \bigl( 2 y_{1}, \frac{y_{2}}{3} \bigr)$}{$\frac{2}{3} f_{1} \bigl( 2 y_{1}, \frac{y_{2}}{3} \bigr)$}
	\end{titwo}

	\begin{titwo}
		设随机变量 $X$ 与 $Y$ 相互独立,且 $X \sim N\bigl( 0,\sigma_{1}^{2} \bigr)$, $Y \sim N\bigl( 0,\sigma_{2}^{2} \bigr)$,则概率 $P \{ |X - Y| < 1 \}$ \kuo.

		\onech{随 $\sigma_{1}$ 的增加而增加,随 $\sigma_{2}$ 的增加而减少}{随 $\sigma_{1}$ 的增加而减少,随 $\sigma_{2}$ 的减少而减少}{随 $\sigma_{1}$ 的增加而减少,随 $\sigma_{2}$ 的减少而增加}{随 $\sigma_{1}$ 的增加而增加,随 $\sigma_{2}$ 的减少而减少}
	\end{titwo}

	\begin{titwo}
		已知二维随机变量 $(X,Y)$ 的概率分布为
		\begin{center}
			\begin{tabular}{c|cc}
				\hline
				\diagbox{$X$}{$Y$} & $0$ & $1$ \\
				\hline
				$0$ & $a$ & $0.4$ \\
				$1$ & $0.1$ & $b$ \\
				\hline
			\end{tabular}
		\end{center}
		若随机事件 $\{ X = 0 \}$ 与 $\{ X + Y = 1 \}$ 相互独立,令 $U = \max \{ X, Y \}$, $V = \min \{ X, Y \}$,则 $P \{ U + V = 1 \} = $ \kuo.

		\fourch{$0.1$}{$0.3$}{$0.5$}{$0.7$}
	\end{titwo}

	\begin{titwo}
		设随机变量 $X$ 与 $Y$ 相互独立,且 $X \sim N(0,1)$, $Y \sim B(n,p)(0 < p < 1)$,则 $X + Y$ 的分布函数 \kuo.

		\twoch{为连续函数}{恰有 $n + 1$ 个间断点}{恰有 $1$ 个间断点}{有无穷多个间断点}
	\end{titwo}

	\begin{titwo}
		设随机变量 $X$ 与 $Y$ 相互独立,且 $X \sim N \bigl( \mu_{1},\sigma_{1}^{2} \bigr)$, $Y \sim N\bigl( \mu_{2},\sigma_{2}^{2} \bigr)$,若 $P\{ X > Y \} < \frac{1}{2}$,则 \kuo.

		\fourch{$\mu_{1} < \mu_{2}$}{$\mu_{1} > \mu_{2}$}{$\sigma_{1} < \sigma_{2}$}{$\sigma_{1} > \sigma_{2}$}
	\end{titwo}

	\begin{titwo}
		设二维随机变量 $(X,Y)$ 在区域 $D = \bigl\{ (x,y) \bigl| 1 \leq x \leq \ee^{2}, 0 \leq y \leq \frac{1}{x} \bigr\}$ 上服从均匀分布,则 $(X,Y)$ 关于 $X$ 的边缘概率密度 $f_{X}(x)$ 在点 $x = \ee$ 处的值为 \htwo.
	\end{titwo}

	\begin{titwo}
		设二维随机变量 $(X,Y)$ 的概率密度为 $f(x,y) = \begin{cases}
			4xy \ee^{-( x^{2} + y^{2} )}, & x > 0, y > 0, \\
			0, & \text{其他},
		\end{cases}$ 则对 $x > 0$, $f_{Y|X} (y|x) = $ \htwo.
	\end{titwo}

	\begin{titwo}
		设二维随机变量 $(X,Y)$ 在 $G = \bigl\{ (x,y) \bigl| - \frac{1}{2} < x < 0, 0 < y < 2x + 1 \bigr\}$ 上服从均匀分布,则条件概率 $P \bigl\{ - \frac{1}{4} < X < 0 \bigl| \frac{1}{2} < Y \leq 1 \bigr\} = $ \htwo.
	\end{titwo}

	\begin{titwo}
		设二维随机变量的分布律为
		\begin{center}
			\begin{tabular}{c|ccc}
				\hline
				\diagbox{$X$}{$Y$} & 1 & 2 & 3 \\
				\hline
				0 & $\frac{1}{6}$ & $\frac{1}{9}$ & $\frac{1}{18}$ \\
				1 & $\frac{1}{3}$ & $\frac{2}{9}$ & $\frac{1}{9}$ \\
				\hline
			\end{tabular}
		\end{center}
		则随机变量 $Z = Y \cdot \min\{ X,Y \}$ 的分布律为 \htwo.
	\end{titwo}

	\begin{titwo}
		设随机变量 $X$ 与 $Y$ 相互独立,且都服从参数为 $1$ 的指数分布,则随机变量 $Z = \frac{Y}{X}$ 的概率密度为 \htwo.
	\end{titwo}

	\begin{titwo}
		设二维随机变量 $(X,Y)$ 的概率密度为 $f(x,y)$,则随机变量 $(2X,Y+1)$ 的概率密度 $f_{1}(x,y) = $ \htwo.
	\end{titwo}

	\begin{titwo}
		已知随机变量 $X_{1}$ 与 $X_{2}$ 的概率分布,
		\[
			X_{1} \sim \begin{psmallmatrix}
				-1 & 0 & 1 \\
				\frac{1}{4} & \frac{1}{2} & \frac{1}{4}
			\end{psmallmatrix},
			X_{2} \sim \begin{psmallmatrix}
				0 & 1 \\
				\frac{1}{2} & \frac{1}{2}
			\end{psmallmatrix},
		\]
		且 $P\{ X_{1} X_{2} = 0 \} = 1$.
		\begin{enumerate}
			\item 求 $X_{1}$ 与 $X_{2}$ 的联合分布;
			\item 问 $X_{1}$ 与 $X_{2}$ 是否独立?为什么?
		\end{enumerate}
	\end{titwo}

	\begin{titwo}
		设 $X$, $Y$ 是相互独立的随机变量,它们都服从参数为 $n$, $p$ 的二项分布,证明:$Z = X + Y$ 服从参数为 $2n$, $p$ 的二项分布.
	\end{titwo}

	\begin{titwo}
		设 $\xi$, $\eta$ 是相互独立且服从同一分布的两个随机变量,已知 $\xi$ 的分布律为 $P\{ \xi = i \} = \frac{1}{3}$, $i = 1$, $2$, $3$,又设 $X = \max \{ \xi,\eta \}$, $Y = \min \{ \xi,\eta \}$,试写出二维随机变量 $(X,Y)$ 的分布律及边缘分布律,并求 $P\{ \xi = \eta \}$.
	\end{titwo}

	\begin{titwo}
		设 $(X,Y)$ 的概率密度为
		\[
			f(x,y) = \begin{cases}
				\ee^{-(x + y)}, & x \geq 0, y \geq 0, \\
				0, & \text{其他},
			\end{cases}
		\]
		问 $X$, $Y$ 是否独立?
	\end{titwo}

	\begin{titwo}
		设 $X$ 关于 $Y$ 的条件概率密度为
		\[
			f_{X|Y}(x|y) = \begin{cases}
				\frac{3x^{2}}{y^{3}}, & 0 < x < y, \\
				0, & \text{其他},
			\end{cases}
		\]
		而 $Y$ 的概率密度为
		\[
			f_{Y}(y) = \begin{cases}
				5y^{4}, & 0 < y < 1, \\
				0, & \text{其他},
			\end{cases}
		\]
		求 $P \bigl\{ X > \frac{1}{2} \bigr\}$.
	\end{titwo}

	\begin{titwo}
		设 $(X,Y)$ 服从 $G = \bigl\{ (x,y) | x^{2} + y^{2} \leq 1 \bigr\}$ 上的均匀分布,试求给定 $Y = y$ 的条件下 $X$ 的条件概率密度 $f_{X|Y}(x|y)$.
	\end{titwo}

	\begin{titwo}
		设随机变量 $X$ 与 $Y$ 相互独立,概率密度分别为
		\[
			f_{X}(x) = \begin{cases}
				1, & 0 < x < 1, \\
				0, & \text{其他},
			\end{cases}
			f_{Y}(y) = \begin{cases}
				\ee^{-y}, & y > 0, \\
				0, & \text{其他},
			\end{cases}
		\]
		求随机变量 $Z = 2X + Y$ 的概率密度 $f_{Z}(z)$.
	\end{titwo}

	\begin{titwo}
		设随机变量 $(X,Y)$ 的概率密度为
		\[
			f(x,y) = \begin{cases}
				3x, & 0 < x < 1, 0 < y < x, \\
				0, & \text{其他},
			\end{cases}
		\]
		求随机变量 $Z = X - Y$ 的概率密度 $f_{Z}(z)$.
	\end{titwo}

	\begin{titwo}
		设二次方程 $x^{2} - Xx + Y = 0$ 的两个根相互独立,且都服从 $(0,2)$ 上的均匀分布,分别求 $X$ 与 $Y$ 的概率密度.
	\end{titwo}

	\begin{titwo}
		设随机变量 $X$ 与 $Y$ 相互独立,都服从均匀分布 $U(0,1)$. 求 $Z = |X - Y|$ 的概率密度及 $P\bigl\{ - \frac{1}{2} < X - Y < \frac{1}{2} \bigr\}$.
	\end{titwo}

	\begin{titwo}
		设随机变量 $(X,Y)$ 的概率密度为
		\[
			f(x,y) = \frac{1}{2 \uppi \sigma^{2}} \ee^{ - \frac{ x^{2} + y^{2} }{ 2\sigma^{2} } }, -\infty < x,y < +\infty,
		\]
		求 $Z = X^{2} + Y^{2}$ 的概率密度 $f_{Z}(z)$.
	\end{titwo}

	\begin{titwo}
		设 $(X,Y)$ 的联合概率密度为
		\[
			f(x,y) = \begin{cases}
				\frac{1}{2}, & -1 \leq x \leq 1, 0 \leq y \leq 1, \\
				0, & \text{其他}.
			\end{cases}
		\]
		求:
		\begin{enumerate}
			\item $Z = |X| + Y$ 的概率密度 $f_{Z}(z)$;
			\item $EZ$.
		\end{enumerate}
	\end{titwo}

	\begin{titwo}
		设随机变量 $X_{1}$, $X_{2}$, $\cdots$, $X_{n}$ 相互独立,且 $X_{i}$ 服从参数为 $\lambda_{i}$ 的指数分布,其概率密度为
		\[
			f_{i}(x) = \begin{cases}
				\lambda_{i} \ee^{- \lambda_{i} x}, & x > 0, \\
				0, & \text{其他},
			\end{cases} i = 1,2,\cdots,n,
		\]
		求 $P\{ X_{1} = \min\{ X_{1},X_{2},\cdots,X_{n} \} \}$.
	\end{titwo}

	\begin{titwo}
		设 $X_{1} \sim P(\lambda_{1})$, $X_{2} \sim P(\lambda_{2})$,且 $X_{1}$ 与 $X_{2}$ 相互独立.
		\begin{enumerate}
			\item 证明 $X_{1} + X_{2}$ 的分布为 $P(\lambda_{1} + \lambda_{2})$;
			\item 求在 $X_{1} + X_{2} = n (n \geq 1)$ 的条件下,$X_{1}$ 的条件分布.
		\end{enumerate}
	\end{titwo}

	\begin{titwo}
		设二维正态随机变量 $(X,Y)$ 的概率密度为 $f(x,y)$,已知条件概率密度 $f_{X|Y}(x|y) = A \* \ee^{ - \frac{2}{3} (x - \frac{y}{2})^{2} }$ 和 $f_{Y|X}(y|x) = B \* \ee^{-\frac{2}{3} (y - \frac{x}{2})^{2} }$. 求:
		\begin{enumerate}
			\item 常数 $A$ 和 $B$;
			\item 边缘概率密度 $f_{X}(x)$ 和 $f_{Y}(x)$;
			\item $f(x,y)$.
		\end{enumerate}
	\end{titwo}

	\begin{titwo}
		设随机变量 $X$, $Y$ 相互独立,且 $P\{ X = 0 \} = P\{ X = 1 \} = \frac{1}{2}$, $P\{ Y \leq x \} = x$, $0 < x \leq 1$. 求 $Z = XY$ 的分布函数.
	\end{titwo}