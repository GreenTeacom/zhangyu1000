\section{多元函数微分学}
	\subsection{概念}

	\begin{ti}
		设 $f(x,y) = \ee^{x + y} \Bigl[ x^{\frac{1}{3}} (y - 1)^{\frac{1}{3}} + y^{\frac{1}{3}} (x - 1)^{\frac{2}{3}} \Bigr]$,则在点 $(0,1)$ 处的两个偏导数 $f_{x}'(0,1)$ 和 $f_{y}'(0,1)$ 的情况为\kuo.

		\onech{两个偏导数均不存在}{$f_{x}'(0,1)$ 不存在,$f_{y}'(0,1) = \frac{4}{3}\ee$}{$f_{x}'(0,1) = \frac{\ee}{3}$,$f_{y}'(0,1) = \frac{4}{3}\ee$}{$f_{x}'(0,1) = \frac{\ee}{3}$,$f_{y}'(0,1)$ 不存在}
	\end{ti}

	\begin{ti}
		函数 $z = f(x,y) = \sqrt{|xy|}$ 在点 $(0,0)$ \kuo.

		\onech{连续,但偏导数不存在}{偏导数存在,但不可微}{可微}{偏导数存在且连续}
	\end{ti}

	\begin{ti}
		函数 $f(x,y) = \sqrt[3]{x^{2} y}$ 在点 $(0,0)$ 处:
		\begin{enumerate}
			\item 是否连续,说明理由;
			\item 偏导数是否存在,说明理由;
			\item 是否可微,说明理由.
		\end{enumerate}
	\end{ti}

	\begin{ti}
		设
		\begin{enumerate}
			\item $f(x,y) = \begin{cases}
				\frac{x^{2} y^{2}}{\left( x^{2} + y^{2} \right)^{3/2}}, & (x,y) \ne (0,0),\\
				0, & (x,y) = (0,0);
			\end{cases}$
			\item $g(x,y) = \begin{cases}
				\bigl( x^{2} + y^{2} \bigr) \sin \frac{1}{x^{2} + y^{2}}, & (x,y) \ne (0,0),\\
				0, & (x,y) = (0,0).
			\end{cases}$
		\end{enumerate}
		讨论它们在点 $(0,0)$ 处的
		\begin{enumerate}
			\item[\libcirc{1}] 偏导数的存在性;
			\item[\libcirc{2}] 函数的连续性;
			\item[\libcirc{3}] 方向导数的存在性;
			\item[\libcirc{4}] 函数的可微性.
		\end{enumerate}
	\end{ti}

	\begin{ti}
		已知 $f(x,y) = \bigl( xy + xy^{2} \bigr) \ee^{x + y}$,则 $\frac{\partial^{10}f}{\partial x^{5} \partial y^{5}} = $\htwo

		\noindent\hone{4}.
	\end{ti}

	\begin{ti}
		\begin{enumerate}
			\item 设 $y = \frac{1}{x(1 - x)}$,求 $\frac{\dd^{n}y}{\dd{x^{n}}}$;
			\item 设 $z = \frac{y^{2}}{x(1 - x)}$,求 $\frac{\partial^{n}z}{\partial x^{n}}$.
		\end{enumerate}
	\end{ti}

	\begin{ti}
		设 $z = y^{2} \ln \bigl( 1 - x^{2} \bigr)$,求 $\frac{\partial^{n}z}{\partial x^{n}}$.
	\end{ti}

	\begin{ti}
		设 $z = x \ln \bigl[ \bigl( 1 + y^{2} \bigr) \ee^{x^{2} \sin y} \bigr]$,则 $\frac{\partial^{4}z}{\partial y^{2} \partial x^{2}} = $\htwo

		\noindent\hone{4}.
	\end{ti}

	\begin{ti}
		设函数 $f(x,y)$ 的一阶偏导数连续,在点 $(1,0)$ 的某邻域内有
		\[
			f(x,y) = 1 - x - 2y + o\left( \sqrt{(x - 1)^{2} + y^{2}} \right)
		\]
		成立. 记 $z(x,y) = f\bigl( \ee^{y}, x + y \bigr)$,则 $\dd{[z(x,y)]}|_{(0,0)} = $\htwo

		\noindent\hone{4}.
	\end{ti}

	\begin{ti}
		设函数 $f(x,y)$ 及它的二阶偏导数在全平面连续,且 $f(0,0) = 0$,$\Bigl| \frac{\partial f}{\partial x} \Bigr| \leq 2 \bigl|x - y\bigr|$,$\Bigl| \frac{\partial f}{\partial y} \Bigr| \leq 2 \bigl|x - y\bigr|$. 求证:$\bigl|f(5,4)\bigr| \leq 1$.
	\end{ti}

	\begin{ti}
		二元函数 $f(x,y) = x^{y}$ 在点 $(\ee,0)$ 处的二阶(即 $n = 2$) 泰勒展开式(不要求写出余项)为\htwo.
	\end{ti}