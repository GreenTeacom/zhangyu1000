\subsection{计算}
	
	\begin{ti}
		计算 $I = \int_{1}^{2} \dd{x} \int_{\frac{1}{x}}^{1} y \ee^{xy} \dd{y}$.
	\end{ti}

	\begin{ti}
		\[
			\int_{0}^{1} \dd{y} \int_{0}^{1} \sqrt{\ee^{2x} - y^{2}} \dd{x} + \int_{1}^{\ee} \dd{y} \int_{\ln y}^{1} \sqrt{\ee^{2x} - y^{2}} \dd{x} = 
		\]
		\kuo.

		\twoch{$\frac{\uppi}{8}\bigl( \ee^{2} - 1 \bigr)$}{$\frac{\uppi}{8}\bigl( \ee^{2} + 1 \bigr)$}{$\frac{\uppi}{4}\bigl( \ee^{2} - 1 \bigr)$}{$\frac{\uppi}{4}\bigl( \ee^{2} + 1 \bigr)$}
	\end{ti}

	\begin{ti}
		已知
		\[
			I = \int_{0}^{2} \dd{x} \int_{0}^{\frac{x^{2}}{2}} f(x,y) \dd{y} + \int_{2}^{2\sqrt{2}} \dd{x} \int_{0}^{\sqrt{8 - x^{2}}} f(x,y) \dd{y},
		\]
		则 $I = $\kuo.

		\onech{$\int_{0}^{2} \dd{y} \int_{\sqrt{2y}}^{\sqrt{8 - y^{2}}} f(x,y) \dd{x}$}{$\int_{0}^{2} \dd{y} \int_{1}^{\sqrt{8 - y^{2}}} f(x,y) \dd{x}$}{$\int_{0}^{1} \dd{y} \int_{\sqrt{2y}}^{\sqrt{8 - y^{2}}} f(x,y) \dd{x}$}{$\int_{0}^{2} \dd{y} \int_{\sqrt{2y}}^{1} f(x,y) \dd{x}$}
	\end{ti}

	\begin{ti}
		累次积分 $\int_{0}^{2R} \dd{y} \int_{0}^{\sqrt{2Ry - y^{2}}} f \bigl( x^{2} + y^{2} \bigr) \dd{x} (R > 0)$ 化为极坐标形式的累次积分为\kuo.

		\onech{$\int_{0}^{\uppi} \dd{\theta} \int_{0}^{2R\sin\theta} f \bigl( r^{2} \bigr) r \dd{r}$}{$\int_{0}^{\frac{\uppi}{2}} \dd{\theta} \int_{0}^{2R\cos\theta} f \bigl( r^{2} \bigr) r \dd{r}$}{$\int_{0}^{\frac{\uppi}{2}} \dd{\theta} \int_{0}^{2R\sin\theta} f \bigl( r^{2} \bigr) r \dd{r}$}{$\int_{0}^{\uppi} \dd{\theta} \int_{0}^{2R\cos\theta} f \bigl( r^{2} \bigr) r \dd{r}$}
	\end{ti}

	\begin{ti}
		计算 $\int_{0}^{1} \dd{y} \int_{\arcsin y}^{\frac{\uppi}{2}} \cos x \cdot \sqrt{1 + \cos^{2}x} \dd{x}$.
	\end{ti}

	\begin{ti}
		计算 $\int_{0}^{1} \dd{y} \int_{3y}^{3} \ee^{x^{2}} \dd{x}$.
	\end{ti}

	\begin{ti}
		计算 $\int_{0}^{1} \dd{y} \int_{\sqrt{y}}^{1} \sqrt{x^{3} + 1} \dd{x}$.
	\end{ti}

	\begin{ti}
		计算 $\int_{0}^{1} \dd{x} \int_{x^{2}}^{1} x^{3} \sin y^{3} \dd{y}$.
	\end{ti}

	\begin{ti}
		计算 $\int_{0}^{1} \dd{x} \int_{x^{2}}^{x} \bigl( x^{2} + y^{2} \bigr)^{-\frac{1}{2}} \dd{y}$.
	\end{ti}

	\begin{ti}
		计算 $\int_{1}^{2} \dd{x} \int_{0}^{x} \frac{y \sqrt{x^{2} + y^{2}}}{x} \dd{y}$.
	\end{ti}

	\begin{ti}
		计算 $\int_{1}^{2} \dd{x} \int_{\sqrt{x}}^{x} \sin \frac{\uppi x}{2y} \dd{y} + \int_{2}^{4} \dd{x} \int_{\sqrt{x}}^{2} \sin \frac{\uppi x}{2y} \dd{y}$.
	\end{ti}

	\begin{ti}
		$\int_{0}^{1} \dd{y} \int_{y}^{1} \Bigl( \frac{\ee^{x^{2}}}{x} - \ee^{y^{2}} \Bigr) \dd{x} = $\htwo.
	\end{ti}

	\begin{ti}
		计算 $\iint_{D} \ee^{\frac{y}{x + y}} \dd{\sigma}$,其中 $D = \bigl\{ (x,y) \bigl| 0 \leq y \leq 1 - x, y \leq x \bigr\}$.
	\end{ti}

	\begin{ti}
		设平面区域 $D = \Bigl\{ (x,y) \Bigl| x^{2} + y^{2} \leq 8, y \geq \frac{x^{2}}{2} \Bigr\}$,计算
		\[
			I = \iint_{D} \bigl[ (x - 1)^{2} + y^{2} \bigr] \dd{\sigma}.
		\]
	\end{ti}

	\begin{ti}
		计算
		\[
			I = \iint_{D} \bigl( x^{2} + xy \bigr)^{2} \dd{x} \dd{y},
		\]
		其中 $D = \bigl\{ (x,y) \bigl| x^{2} + y^{2} \leq 2x \bigr\}$.
	\end{ti}

	\begin{ti}
		计算 $I = \iint_{\sqrt{x} + \sqrt{y} \leq 1} \sqrt[3]{\sqrt{x} + \sqrt{y}} \dd{x} \dd{y}$.
	\end{ti}

	\begin{ti}
		设函数 $f(x,y)$ 连续,且
		\[
			f(x,y) = x + \iint_{D} y f(u,v) \dd{u} \dd{v},
		\]
		其中 $D$ 由 $y = \frac{1}{x}$,$x = 1$,$y = 2$ 围成,求 $f(x,y)$.
	\end{ti}

	\begin{ti}
		设 $D = \bigl\{ (x,y) \bigl| |x| \leq 2, |y| \leq 2 \bigr\}$,计算
		\[
			I = \iint_{D} \bigl| x^{2} + y^{2} - 1 \bigr| \dd{\sigma}.
		\]
	\end{ti}

	\begin{ti}
		设 $D = \bigl\{ (x,y) \bigl| 0 \leq x \leq 1, 0 \leq y \leq 2\ee \bigr\}$,计算
		\[
			\iint_{D} x \bigl| y - \ee^{x} \bigr| \dd{\sigma}.
		\]
	\end{ti}

	\begin{ti}
		计算 $I = \iint_{D} \bigl( |x| + |y| \bigr) \dd{x} \dd{y}$,其中 $D$ 是由曲线 $xy = 2$,直线 $y = x - 1$ 及 $y = x + 1$ 所围成的区域.
	\end{ti}

	\begin{ti}
		设 $D = \bigl\{ (x,y) \bigl| 0 \leq x \leq \uppi, 0 \leq y \leq 2 \bigr\}$,计算 $\iint_{D} \bigl| y - \sin x \bigr| \dd{\sigma}$.
	\end{ti}

	\begin{ti}
		计算
		\[
			I = \int_{-1}^{1} \dd{x} \int_{x}^{2 - |x|} \bigl[ \ee^{|y|} + \sin \bigl( x^{3}y^{3} \bigr) \bigr] \dd{y}.
		\]
	\end{ti}

	\begin{ti}
		设 $f(x,y) = \begin{cases}
			1 - x - y, & x + y \leq 1,\\
			2, & x + y > 1,
		\end{cases}$ 计算
		\[
			\iint_{D} f(x,y) \dd{x} \dd{y},
		\]
		其中 $D$ 为正方形区域 $\bigl\{ (x,y) \bigl| 0 \leq x \leq 1, 0 \leq y \leq 1 \bigr\}$.
	\end{ti}

	\begin{ti}
		设函数 $f(x) = \begin{cases}
			x, & 0 \leq x \leq 2,\\
			0, & x < 0 \text{\ 或\ } x > 2,
		\end{cases}$ 计算 $I = \iint_{D} \frac{f(x + y)}{f\left( \sqrt{x^{2} + y^{2}} \right)} \dd{x} \dd{y}$,其中 $D = \bigl\{ (x,y) \bigl| x^{2} + y^{2} \leq 4 \bigr\}$.
	\end{ti}

	\begin{ti}
		计算 $\iint_{D} \min\bigl\{ x,y \bigr\} \dd{x} \dd{y}$,其中 $D = \bigl\{ (x,y) \bigl| 0 \leq x \leq 3, 0 \leq y \leq 1 \bigr\}$.
	\end{ti}

	\begin{ti}
		计算 $\int_{0}^{a} \dd{x} \int_{0}^{b} \ee^{ \max\left\{ b^{2}x^{2}, a^{2}y^{2} \right\} } \dd{y}$,其中 $a,b > 0$.
	\end{ti}

	\begin{ti}
		设 $F(x,y) = \frac{\partial^{2}f(x,y)}{\partial x \partial y}$ 在 $D = [a,b] \times [c,d]$ 上连续,求
		\[
			I = \iint_{D} F(x,y) \dd{x} \dd{y},
		\]
		并证明:$I \leq 2(M - m)$,其中 $M$ 和 $m$ 分别是 $f(x,y)$ 在 $D$ 上的最大值和最小值.
	\end{ti}

	\begin{ti}
		设函数 $f(x)$ 在 $[0,1]$ 上连续,证明:
		\[
			\int_{0}^{1} \ee^{f(x)} \dd{x} \int_{0}^{1} \ee^{-f(y)} \dd{y} \geq 1.
		\]
	\end{ti}

	\begin{ti}
		设 $f(x,y)$ 为连续函数,则
		\[
			I = \lim_{t \to 0^{+}} \frac{1}{\uppi t^{2}} \iint_{D} f(x,y) \dd{\sigma}
		\] = \htwo,其中 $D = \bigl\{ (x,y) \bigl| x^{2} + y^{2} \leq t^{2} \bigr\}$.
	\end{ti}

	\begin{ti}
		已知 $f(t) = \iint_{D(t): x^{2} + y^{2} \leq t^{2}} \bigl( \ee^{x^{2} + y^{2}} - ky^{2} \bigr) \dd{\sigma}$ 在 $t \in (0,+\infty)$ 内是单调增加函数,$k$为常数,求 $k$ 的最大取值范围.
	\end{ti}

	\begin{ti}
		由曲线 $y = x^{2}$,$y = x + 2$ 所围成的平面薄片,其上各点处的面密度 $\mu = 1 + x^{2}$,则此薄片的质量 $M = $\htwo.
	\end{ti}

	\begin{ti}
		求柱体 $x^{2} + y^{2} \leq 2x$ 被 $x^{2} + y^{2} + z^{2} = 4$ 所截得部分的体积.
	\end{ti}

	\begin{ti}
		设平面薄片所占的区域 $D$ 由抛物线 $y = x^{2}$ 及直线 $y = x$ 所围成,它在 $(x,y)$ 处的面密度 $\rho(x,y) = x^{2}y$,求此薄片的重心.
	\end{ti}