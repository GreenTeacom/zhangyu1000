\section{一元函数积分学}
	\subsection{概念与性质}

	\begin{ti}
		设 $f(x)$ 为连续函数,$F(x) = \int_{0}^{x} f(t) \dd{t}$. 试证明:
		\begin{enumerate}
			\item $F(x)$ 的奇偶性正好与 $f(x)$ 的奇偶性相反;
			\item 若 $f(x)$ 为奇函数,则 $f(x)$ 的一切原函数均为偶函数; 若 $f(x)$ 为偶函数,则有且仅有一个原函数为奇函数.
		\end{enumerate}
	\end{ti}

	\begin{ti}
		设 $f(x)$ 在 $(-\infty,+\infty)$ 内连续,以 $T$ 为周期,证明:
		\begin{enumerate}
			\item $\int_{a}^{a + T} f(x) \dd{x} = \int_{0}^{T} f(x) \dd{x}$ ($a$ 为任意实数);
			\item $\int_{0}^{x} f(t) \dd{t}$ 以 $T$ 为周期 $\Longleftrightarrow$ $\int_{0}^{T} f(x) \dd{x} = 0$;
			\item $\int f(x) \dd{x}$ ($f(x)$ 的全体原函数) 周期为 $T$ $\Longleftrightarrow$ $\int_{0}^{T} f(x) \dd{x} = 0$.
		\end{enumerate}
	\end{ti}

	\begin{ti}
		设 $f(x)$ 连续,则在下列变上限积分中,必为偶函数的是\kuo.

		\twoch{$\int_{0}^{x} t \bigl[ f(t) + f(-t) \bigr] \dd{t}$}{$\int_{0}^{x} t \bigl[ f(t) - f(-t) \bigr] \dd{t}$}{$\int_{0}^{x} f\left( t^{2} \right) \dd{t}$}{$\int_{0}^{x} f^{2}(t) \dd{t}$}
	\end{ti}

	\begin{ti}
		设 $F(x) = \int_{x}^{x + 2\uppi} \ee^{\sin t} \dd{t}$,则 $F(x)$ \kuo.

		\twoch{为正常数}{为负常数}{恒为零}{不为常数}
	\end{ti}

	\begin{ti}
		设 $f(x)$ 是以 $l$ 为周期的周期函数,则
		\[
			\int_{a + kl}^{a + (k + 1)l} f(x) \dd{x}
		\]
		的值\kuo.

		\twoch{仅与 $a$ 有关}{仅与 $a$ 无关}{与 $a$ 及 $k$ 都无关}{与 $a$ 及 $k$ 都有关}
	\end{ti}

	\begin{ti}
		设 $f(x)$ 是以 $T$ 为周期的可微函数,则下列函数中以 $T$ 为周期的函数是\kuo.

		\twoch{$\int_{a}^{x} f(t) \dd{t}$}{$\int_{a}^{x} f\bigl( t^{2} \bigr) \dd{t}$}{$\int_{a}^{x} f'\bigl( t^{2} \bigr) \dd{t}$}{$\int_{a}^{x} f(t) f'(t) \dd{t}$}
	\end{ti}

	\begin{ti}
		设 $f(x)$ 是以 $2$ 为周期的连续函数,$G(x) = 2 \int_{0}^{x} f(t) \dd{t} - x \int_{0}^{2} f(t) \dd{t}$,则\kuo.

		\onech{$G(x)$ 是以 $2$ 为周期的周期函数,$G'(x)$ 也是以 $2$ 为周期的周期函数}{$G(x)$ 是以 $2$ 为周期的周期函数,$G'(x)$ 不是以 $2$ 为周期的周期函数}{$G(x)$ 不是以 $2$ 为周期的周期函数,$G'(x)$ 是以 $2$ 为周期的周期函数}{$G(x)$ 不是以 $2$ 为周期的周期函数,$G'(x)$ 也不是以 $2$ 为周期的周期函数}
	\end{ti}

	\begin{ti}
		设 $f(x)$ 可导,且
		\[
			f(x) = x + x \int_{0}^{1} f(x) \dd{x} + x^{2} \lim_{x \to 0} \frac{f(x)}{x},
		\]
		求 $f(x)$.
	\end{ti}

	\begin{ti}
		已知 $f(x)$ 在 $[-1,1]$ 上连续,
		\[
			f(x) = 3x - \sqrt{1 - x^{2}} \int_{0}^{1} f^{2}(x) \dd{x},
		\]
		求 $f(x)$.
	\end{ti}